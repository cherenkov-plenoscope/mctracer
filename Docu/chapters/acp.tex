\chapter{The \acf{acp}}
%
The \tool{} can simulate \acp{acp}.
%
An \ac{acp} consists out of two main parts.
%
First, an imaging system like e.g. a segmented imaging reflector as it is used for classic \acp{iact}.
%
Second, a light field sensor.
%
%------------------------------------------------------------------------------
\section{Create an \ac{acp} scenery}
%
Lets create an example scenery of an \ac{acp} called PLERITAS, which is a plenoptic extension to the VERITAS \ac{iact}.
%
\begin{lstlisting}[style=MctBash]
/demo$ mkdir pleritas
/demo$ cd pleritas/
\end{lstlisting}
All the resources needed to describe PLERITAS have to be in a folder.
%
\begin{lstlisting}[style=MctBash]
demo/pleritas$ vi scenery.xml
\end{lstlisting}
%
Create a xml file called scenery.xml and describe your scenery in there.
%
For our PLERITAS we use a basic VERITAS like imaging reflecor (created using the segmented reflector tool),\\
%
\XmlFileStartEnd{demo/pleritas/scenery.xml}{reflector_s}{stl_s}
%
a light field sensor\\
%
\XmlFileStartEnd{demo/pleritas/scenery.xml}{light_field_sensor_s}{light_field_sensor_e}
%
and an additional spider web which causes additional shadowing and is described in a CAD file.\\
%
\XmlFileStartEnd{demo/pleritas/scenery.xml}{stl_s}{light_field_sensor_s}
%
\newline
\SideBySide{
\CenFig{demo/pleritas_reflector.png}{1.0}	
}{
\CenFig{demo/pleritas_light_field_sensor.png}{1.0}	
}
%
\SideBySide{
\CenFig{demo/pleritas_focus_reflector.png}{1.0}	
}{
\CenFig{demo/sensor_back.png}{1.0}	
}
%
All resources like the CAD file of the spider web must be placed in the scenery folder.
%
\begin{lstlisting}[style=MctBash]
/demo/pleritas$ ls
scenery.xml  spider.stl
\end{lstlisting}
%

Explore your scenery using mctShow.
%
\begin{lstlisting}[style=MctBash]
/demo/pleritas$ mctShow -s scenery.xml
\end{lstlisting}
%
%------------------------------------------------------------------------------
\section{Run the light field calibration}
%
\begin{lstlisting}[style=MctBash]
/demo$ mctPlenoscopeCalibration -i scenery -o pleritas_calibration -n 3
Plenoscope Calibrator: propagating 3M photons
1 of 3
2 of 3
3 of 3
/demo$ 
\end{lstlisting}
%
\CppFileStartEnd{../Plenoscope/Calibration/LixelStatistics.h}{lixel_statistics_s}{lixel_statistics_e}
%
%------------------------------------------------------------------------------
\section{Simulate \ac{acp} responses to \ac{eas}}
%
\begin{lstlisting}[style=MctBash]
/demo$ mctPlenoscopePropagation -c propagation_config.xml -l pleritas_calibration -i /some/simulation/gamma1_RUN41.dat -o my_run
event 1, PRMPAR 1, E 225.569 GeV
event 2, PRMPAR 1, E 168.365 GeV
event 3, PRMPAR 1, E 46.5717 GeV
...
\end{lstlisting}
%------------------------------------------------------------------------------
\section{Explore the siumlated events}

\section{Light Field calibration}
%
Each read out channel (lixel) on the light field sensor of the \ac{acp} corresponds to a specific ray in the light field. 
%
Each of these rays has a support on the principal aperture plane at position $x$ and $y$ on and a direction vector descibed by $c_x$ and $c_y$.
%
Further, each of these lixel has a specific time delay $t_\text{delay}$ which the light needs to travel when comig from the principal aperture plane and each lixel has its own efficency $\eta$ due to its specific geometrical position in the set up.
%
So in the light field calibration we determine $\eta$, $x$, $y$, $c_x$, $c_y$ and $t_\text{delay}$ for each lixel.
%
The calibration is done by throughing photons into the \ac{acp}, where we randomly draw both the photons intersection on the principal aperture plane $x$, $y$ and their incoming direction $c_x$, $c_y$.
%
In the calibration, many photons are used and several of them will be absorbed in the lixels.
%
For each lixel, there is list of the photon properties ($x$, $y$, $c_x$, $c_y$ and $t_\text{delay}$) of the photons that reached this lixel.
%
From this list, the lixel is assigned the averages of all these properties, as well as their standard deviations.
%
The number of photons reaching the lixel during the calibration gives the efficiency $\eta$.