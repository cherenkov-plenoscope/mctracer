\chapter{1D functions}
\newcommand{\la}{\lambda}
%
The \lstinline{Function::Func1D} class provides $1$D mapping for floating numbers.
%
\begin{eqnarray}
    y &=& f(x)\\
    x &\in& X
\end{eqnarray}
%
All functions have limits which need to be respected. Any call of a function $f(x)$ with $x \notin X$ will throw an exception. We are strict about this behaviour to enforce that no propagation passes silently where e.g. your mirror's reflective index is only defined up to $600\,$nm but you shoot $800\,$nm photons onto it. Functions live in their own namespace.
%
\CppFileStartEnd{../Tests/Examples/Func1DExample.cpp}{using_namespace}{using_namespace_end}
%------------------------------------------------------------------------------
\section{Domains and their limits}
%
First we define limits for the domains of our functions.
%
\CppFileStartEnd{../Tests/Examples/Func1DExample.cpp}{func_limits}{func_limits_assert}
%
The limits here include the lower bound $0.0$ and exclude the upper one $1.0$. A limit can assert that a given argument is in its domain. If not, it will throw an exception.
%
\CppFileStartEnd{../Tests/Examples/Func1DExample.cpp}{func_limits_assert}{func_limits_constant}
%
All functions have a domain within their limits. The limits are given to the funcions during construction.
%
\CppFileStartEnd{../Tests/Examples/Func1DExample.cpp}{func_limits_constant}{func_limits_const_call}
%
Functions assert, the argument to be inside their domain.
%
\CppFileStartEnd{../Tests/Examples/Func1DExample.cpp}{func_limits_const_call}{func_limits_call_end}
%------------------------------------------------------------------------------
\section{Constant}
%
Sometimes it needs a constant function which will return the same value for any argument inside their domain limits.
%
\begin{eqnarray}
    y &=& f(x) = c
\end{eqnarray}
%
A constant function is created given its single constant value e.g. $1.337$ and its domain limits.
%
\CppFileStartEnd{../Tests/Examples/Func1DExample.cpp}{func_const}{func_const_call}
%
When called, within the limits, it will always return its constant value.
%
\CppFileStartEnd{../Tests/Examples/set_up_scenery.cpp}{func_const_call}{func_const_call_end}
\CenFig{figures/function_const.png}{0.75}
%------------------------------------------------------------------------------
\section{Polynom3}
%
The versatile polynom to the power of 3 is defined by its four parameters $a,b,c$ and $d$.
%
\begin{eqnarray}
    y &=& f(x) = ax^3 + bx^2 + cx^1 + dx^0
\end{eqnarray}
%
We initialize the \lstinline{Polynom3} using $a,b,c,d$ and the limits. 
%
By setting the higer orders to zero, we create e.g. a linear mapping.
%
\CppFileStartEnd{../Tests/Examples/Func1DExample.cpp}{func_poly3}{func_poly3_call}
\CenFig{figures/function_polynom1.png}{0.75}
%
We can do a quadratic mapping.
\CppFileStartEnd{../Tests/Examples/Func1DExample.cpp}{func_poly3_quad}{func_poly3_quad_end}
\CenFig{figures/function_polynom2.png}{0.75}
%
The full polynom to the power of $3$.
%
\CppFileStartEnd{../Tests/Examples/Func1DExample.cpp}{func_poly3_tri}{func_poly3_tri_end}
\CenFig{figures/function_polynom3.png}{0.75}
%
%------------------------------------------------------------------------------
\section{Linear interpolation look up table}
%
In some cases, it can be tough to model an analytic $1D$ function. In these cases one can still use the a look up table with linear interpolation.
%
The input table also defines the domain limits, so no limits have to be given during construction.
%
\CppFileStartEnd{../Tests/Examples/Func1DExample.cpp}{look_up}{look_up_end}
\CenFig{figures/function_interpol.png}{0.75}
%------------------------------------------------------------------------------
\section{Concatenation}
%
Functions can be concatenated when their domain limits match.
%
The functions to be concatenated can be of any kind, even concatenated functions themselve.
%
Since the concatenated function can deduce its domain limits from the input functions, no limit has to be given during construction.
%
\CppFileStartEnd{../Tests/Examples/Func1DExample.cpp}{func_concat}{func_concat_end}
\CenFig{figures/function_concat.png}{0.75}
%------------------------------------------------------------------------------
\section{Access}
%
Access to the values of a function is done using the bracket operator.
%
\CppFileStartEnd{../Tests/Examples/Func1DExample.cpp}{func_access}{func_access_end}
%
Also a function can provide a table of both argument and value. The number of samples along the domain limits of the function can be specified.
% 
\CppFileStartEnd{../Tests/Examples/Func1DExample.cpp}{func_access_sampling}{func_access_sampling_end}
%
Using the ascci io, a function can be exported into a text file.
%
\CppFileStartEnd{../Tests/Examples/Func1DExample.cpp}{func_access_sampling_export}{func_access_sampling_export_end}
%
The output text file is a two column matrix. First column is the argument $x$, second is the function value $f(x)$.
%
\TxtFile{figures/my_p3.txt}
%------------------------------------------------------------------------------