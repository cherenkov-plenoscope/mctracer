\documentclass[11pt,a4paper,oneside,titlepage]{book}

\usepackage[english]{babel}
\usepackage{lineno,hyperref}
\usepackage[nolist]{acronym}
\usepackage[onehalfspacing]{setspace}
\usepackage{booktabs}
\usepackage{float}
\usepackage{amsmath}
\usepackage{listings}
\usepackage{color}
\usepackage{hyperref}
\usepackage{graphicx}
\usepackage{blindtext}

\definecolor{LightGray}{gray}{0.70}
\lstdefinestyle{MctCpp}{%
    language=C++,
    keywordstyle=\bfseries,
    commentstyle=\itshape,
    rangeprefix=//--,rangesuffix=--,
    includerangemarker=false,
    columns=spaceflexible,
    escapeinside={/*@}{@*/},
    tabsize=4,
    frame=leftline,
    rulecolor=\color{LightGray},
    basicstyle=\ttfamily,
    numbers=left,
    numberstyle=\normalfont\tiny\color{LightGray},
    xleftmargin=0.75cm,
}

\lstdefinestyle{MctTxt}{%
    language={},
    keywordstyle=\bfseries,
    commentstyle=\itshape,
    rangeprefix=//--,rangesuffix=--,
    includerangemarker=false,
    columns=spaceflexible,
    escapeinside={/*@}{@*/},
    tabsize=4,
    frame=leftline,
    rulecolor=\color{LightGray},
    basicstyle=\ttfamily,
    numbers=left,
    numberstyle=\normalfont\tiny\color{LightGray},
    xleftmargin=0.75cm,
}


%\oddsidemargin = 31pt
%\topmargin = 20pt
%\headheight = 12pt
%\headsep = 25pt
\textheight = 25cm
\textwidth = 15cm
%\marginparsep = 10pt
%\marginparwidth = 35pt
%\footskip = 30pt

%\marginparpush = 7pt (not shown)
\hoffset = -1cm
\voffset = -2cm
%\paperwidth = 597pt
%\paperheight = 845pt


%\modulolinenumbers[1]

\begin{document}
%------------------------------------------------------------------------------
\newcommand{\CppFileStartEnd}[3]{%
    \begin{footnotesize}
        \lstinputlisting[linerange=#2-#3, style=MctCpp]{#1}%
    %\hfill\path{#1}\\
    \end{footnotesize}
}

\newcommand{\TxtFile}[1]{%
    \begin{footnotesize}
        \lstinputlisting[style=MctTxt]{#1}%
    \end{footnotesize}
}

\newcommand{\ill}[1]{% In Line Listing
    \begin{lstlisting}#1\end{lstlisting}    
}

\newcommand{\CenFig}[2]{
    \begin{figure}[H]
        \begin{center}
            \includegraphics[width=#2\textwidth]{#1}
        \end{center}
    \end{figure}
}

\makeatletter
\lst@Key{matchrangestart}{f}{\lstKV@SetIf{#1}\lst@ifmatchrangestart}
\def\lst@SkipToFirst{%
    \lst@ifmatchrangestart\c@lstnumber=\numexpr-1+\lst@firstline\fi
    \ifnum \lst@lineno<\lst@firstline
        \def\lst@next{\lst@BeginDropInput\lst@Pmode
        \lst@Let{13}\lst@MSkipToFirst
        \lst@Let{10}\lst@MSkipToFirst}%
        \expandafter\lst@next
    \else
        \expandafter\lst@BOLGobble
    \fi}
\makeatother
%------------------------------------------------------------------------------
\newcommand{\thetitle}{Mooneyes}
%
\thispagestyle{empty}
\begin{center}
\Huge\textbf{\thetitle}
%
\vfill
%
\Large
PhD \\
\vspace{20pt}
\normalsize
submitted by \\[5pt]
{\Large Sebastian Achim M\"uller } \\[5pt]
born in Dortmund Germany, April 5, 1988\\[5pt]
%
{\normalsize sebastian8.mueller@tu-dortmund.de}\\
%
\vspace{20pt}
Institute for Particle Physics
\par\smallskip\noindent
Chair Experimental Physics 5\\Astro Particle Physics\\
\par\smallskip\noindent
ETH Zurich\\2015-2017
\end{center}
%-------------------------------------------------------------
\newpage
\thispagestyle{empty}
%
\newlength{\vertspace}
\setlength{\vertspace}{5pt}
%
\begin{tabular}{p{1.5in} p{2in}}
    1st Examiner & Prof. Dr. Felicitas Pauss\\[\vertspace]
    2nd Examiner & Prof. Dr. Adrian Biland\\[\vertspace]
    Submission date & \dotfill\\ 
\end{tabular}

\newpage
%-------------------------------------------------------------
\pagenumbering{Roman}
\addcontentsline{toc}{chapter}{Contents}
\tableofcontents
%\newpage
%\addcontentsline{toc}{chapter}{Figures}
%\listoffigures
%\newpage
%\addcontentsline{toc}{chapter}{Tables}
%\listoftables
%-------------------------------------------------------------
\cleardoublepage
%
\setcounter{page}{0}
\pagenumbering{arabic}
%
\chapter{abstract}
%
In $\gamma$ ray and cosmic ray astronomy it needs dedicated simulations of detectors to develope and run the instruments which observe particle interactions far beyond any energy accessable in the lab.
%
The Monte Carlo Tracer exists since we do what we must, because we can.
%\linenumbers
%-------------------------------------------------------------------------------
\chapter{The scenery tree}
\blindtext[2]
%-------------------------------------------------------------------------------
\section{The root of the scenery tree}
\label{SubSecRootFrame}
The frame at the root of the tree structure represents the whole scenery.
%
Before ray tracing is performed on the scenery tree, all frames in the tree estimate threir position and orientation w.r.t. the root frame.
%
This way rays can easily and fast be transformed back and forth from the root tree to an individual object frame.
%-------------------------------------------------------------------------------
\chapter{How to set up a scenery in source code}
%
We will build a little scenery of a house with a roof and chimney as well as a simple tree. Further we add a small telescope with a reflective imaging mirror.
%
First we will define the geometry and their surfaces, second we will declare the relations between them. Third and finally we will update all frames relation w.r.t. the root frame to enable fast tracing (post initializing).
% 
%
First we define the main frame of our scenery. The main frame, often called world, will be the root of the scenery tree \ref{SubSecRootFrame}. 
%
\CppFileStartEnd{examples/set_up_scenery.cpp}{world}{tree}
%
Second we define frames that hold individual structures like a tree which will be composed from several objects. The tree will be placed in $x=5\,$m w.r.t. its later mother frame, i.e. the wolrd. 
%
\CppFileStartEnd{examples/set_up_scenery.cpp}{tree}{house}
Also part of the tree is the wooden pole.
\CppFileStartEnd{examples/set_up_scenery.cpp}{leaf_ball}{tree_pole}
and the rest of the source...
\CppFileStartEnd{examples/set_up_scenery.cpp}{tree_pole}{end_set_up_scene_in_source}
\chapter{1D functions}
\newcommand{\la}{\lambda}
%
The \lstinline{Function::Func1D} class provides $1$D mapping for floating numbers.
%
\begin{eqnarray}
    y &=& f(x)\\
    x &\in& X
\end{eqnarray}
%
All functions have limits which need to be respected. Any call of a function $f(x)$ with $x \notin X$ will throw an exception. We are strict about this behaviour to enforce that no propagation passes silently where e.g. your mirror's reflective index is only defined up to $600\,$nm but you shoot $800\,$nm photons onto it. Functions live in their own namespace.
%
\CppFileStartEnd{../Tests/Examples/Func1DExample.cpp}{using_namespace}{using_namespace_end}
%------------------------------------------------------------------------------
\section{Domains and their limits}
%
First we define limits for the domains of our functions.
%
\CppFileStartEnd{../Tests/Examples/Func1DExample.cpp}{func_limits}{func_limits_assert}
%
The limits here include the lower bound $0.0$ and exclude the upper one $1.0$. A limit can assert that a given argument is in its domain. If not, it will throw an exception.
%
\CppFileStartEnd{../Tests/Examples/Func1DExample.cpp}{func_limits_assert}{func_limits_constant}
%
All functions have a domain within their limits. The limits are given to the funcions during construction.
%
\CppFileStartEnd{../Tests/Examples/Func1DExample.cpp}{func_limits_constant}{func_limits_const_call}
%
Functions assert, the argument to be inside their domain.
%
\CppFileStartEnd{../Tests/Examples/Func1DExample.cpp}{func_limits_const_call}{func_limits_call_end}
%------------------------------------------------------------------------------
\section{Constant}
%
Sometimes it needs a constant function which will return the same value for any argument inside their domain limits.
%
\begin{eqnarray}
    y &=& f(x) = c
\end{eqnarray}
%
A constant function is created given its single constant value e.g. $1.337$ and its domain limits.
%
\CppFileStartEnd{../Tests/Examples/Func1DExample.cpp}{func_const}{func_const_call}
%
When called, within the limits, it will always return its constant value.
%
\CppFileStartEnd{../Tests/Examples/set_up_scenery.cpp}{func_const_call}{func_const_call_end}
\CenFig{figures/function_const.png}{0.75}
%------------------------------------------------------------------------------
\section{Polynom3}
%
The versatile polynom to the power of 3 is defined by its four parameters $a,b,c$ and $d$.
%
\begin{eqnarray}
    y &=& f(x) = ax^3 + bx^2 + cx^1 + dx^0
\end{eqnarray}
%
We initialize the \lstinline{Polynom3} using $a,b,c,d$ and the limits. 
%
By setting the higer orders to zero, we create e.g. a linear mapping.
%
\CppFileStartEnd{../Tests/Examples/Func1DExample.cpp}{func_poly3}{func_poly3_call}
\CenFig{figures/function_polynom1.png}{0.75}
%
We can do a quadratic mapping.
\CppFileStartEnd{../Tests/Examples/Func1DExample.cpp}{func_poly3_quad}{func_poly3_quad_end}
\CenFig{figures/function_polynom2.png}{0.75}
%
The full polynom to the power of $3$.
%
\CppFileStartEnd{../Tests/Examples/Func1DExample.cpp}{func_poly3_tri}{func_poly3_tri_end}
\CenFig{figures/function_polynom3.png}{0.75}
%
%------------------------------------------------------------------------------
\section{Linear interpolation look up table}
%
In some cases, it can be tough to model an analytic $1D$ function. In these cases one can still use the a look up table with linear interpolation.
%
The input table also defines the domain limits, so no limits have to be given during construction.
%
\CppFileStartEnd{../Tests/Examples/Func1DExample.cpp}{look_up}{look_up_end}
\CenFig{figures/function_interpol.png}{0.75}
%------------------------------------------------------------------------------
\section{Concatenation}
%
Functions can be concatenated when their domain limits match.
%
The functions to be concatenated can be of any kind, even concatenated functions themselve.
%
Since the concatenated function can deduce its domain limits from the input functions, no limit has to be given during construction.
%
\CppFileStartEnd{../Tests/Examples/Func1DExample.cpp}{func_concat}{func_concat_end}
\CenFig{figures/function_concat.png}{0.75}
%------------------------------------------------------------------------------
\section{Access}
%
Access to the values of a function is done using the bracket operator.
%
\CppFileStartEnd{../Tests/Examples/Func1DExample.cpp}{func_access}{func_access_end}
%
Also a function can provide a table of both argument and value. The number of samples along the domain limits of the function can be specified.
% 
\CppFileStartEnd{../Tests/Examples/Func1DExample.cpp}{func_access_sampling}{func_access_sampling_end}
%
Using the ascci io, a function can be exported into a text file.
%
\CppFileStartEnd{../Tests/Examples/Func1DExample.cpp}{func_access_sampling_export}{func_access_sampling_export_end}
%
The output text file is a two column matrix. First column is the argument $x$, second is the function value $f(x)$.
%
\TxtFile{figures/my_p3.txt}
%------------------------------------------------------------------------------
%------------------------------------------------------------------------------
%--references--
\renewcommand{\bibname}{References}
\bibliography{mct}
\bibliographystyle{plain}  
\addcontentsline{toc}{chapter}{\bibname}
%
\begin{acronym}
    \acro{mct}[MCt]{Monte Carlo Tracer}
\end{acronym}
%--Appendix--
\end{document}