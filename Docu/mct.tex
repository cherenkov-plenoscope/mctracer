\documentclass[review]{elsarticle}

\usepackage{lineno,hyperref}
\usepackage[nolist]{acronym}
\usepackage[onehalfspacing]{setspace}
\usepackage{booktabs}
\usepackage{float}
\usepackage{amsmath}
\usepackage{listings}
\usepackage{color}
\usepackage{hyperref}

\definecolor{LightGray}{gray}{0.70}
\lstdefinestyle{MctCpp}{%
    language=C++,
    keywordstyle=\bfseries,
    commentstyle=\itshape,
    rangeprefix=//--,rangesuffix=--,
    includerangemarker=false,
    columns=spaceflexible,
    escapeinside={/*@}{@*/},
    tabsize=4,
    frame=leftline,
    rulecolor=\color{LightGray},
    basicstyle=\ttfamily,
    numbers=left,
    numberstyle=\normalfont\tiny\color{LightGray}
}

\modulolinenumbers[1]
\journal{Journal of \LaTeX\ Templates}
\bibliographystyle{elsarticle-num}

\begin{document}
%-------------------------------------------------------------------------------
\newcommand{\CppFileStartEnd}[3]{%
    \begin{small}
        \lstinputlisting[linerange=#2-#3, style=MctCpp]{#1}%
    %\hfill\path{#1}\\
    \end{small}
}

\newcommand{\ill}[1]{% In Line Listing
    \begin{lstlisting}#1\end{lstlisting}    
}

\newcommand{\CenFig}[1]{
    \begin{figure}[H]
        \begin{center}
            \includegraphics[width=1\textwidth]{#1}
        \end{center}
    \end{figure}
}

\makeatletter
\lst@Key{matchrangestart}{f}{\lstKV@SetIf{#1}\lst@ifmatchrangestart}
\def\lst@SkipToFirst{%
    \lst@ifmatchrangestart\c@lstnumber=\numexpr-1+\lst@firstline\fi
    \ifnum \lst@lineno<\lst@firstline
        \def\lst@next{\lst@BeginDropInput\lst@Pmode
        \lst@Let{13}\lst@MSkipToFirst
        \lst@Let{10}\lst@MSkipToFirst}%
        \expandafter\lst@next
    \else
        \expandafter\lst@BOLGobble
    \fi}
\makeatother
%-------------------------------------------------------------------------------

\begin{frontmatter}

\title{%
	The Monte Carlo Tracer%\tnoteref{mytitlenote}
}
%\tnotetext[mytitlenote]{%
%	Fully documented templates are available in the elsarticle package on \href{http://www.ctan.org/tex-archive/macros/latex/contrib/elsarticle}{CTAN}.
%}

%% Group authors per affiliation:
%\author{The FACT collaboration}%\fnref{myfootnote}}
%\address{Radarweg 29, Amsterdam}
%\fntext[myfootnote]{TU Dortmund, ETH Zuerich}

%% or include affiliations in footnotes:
\author[mymainaddress,mysecondaryaddress]{Sebastian Achim Mueller}
%\ead[url]{www.fact-project.org}

\cortext[mycorrespondingauthor]{Sebastian Achim Mueller}
\ead{sebmuell@phys.ethz.ch}

\address[mymainaddress]{Institute for Particle Physics, ETH, Otto-Stern-Weg 5, 8093 Zuerich, Switzerland}
\address[mysecondaryaddress]{Experimental Physics 5b, TU Dortmund, Otto-Hahn-Strasse 4, 44227 Dortmund, Germany}
%-------------------------------------------------------------------------------
\begin{abstract}
%
In $\gamma$ ray and cosmic ray astronomy it needs dedicated simulations of detectors to develope and run the instruments which observe particle interactions far beyond any energy accessable in the lab.
%
The Monte Carlo Tracer exists since we do what we must, because we can.
%
\end{abstract}
%-------------------------------------------------------------------------------
\begin{keyword}
ray tracing, photon propagation
%
\end{keyword}
%-------------------------------------------------------------------------------
\end{frontmatter}
%\linenumbers
%-------------------------------------------------------------------------------
\section{The scenery tree}
%-------------------------------------------------------------------------------
\subsection{The root of the scenery tree}
\label{SubSecRootFrame}
The frame at the root of the tree structure represents the whole scenery.
%
Before ray tracing is performed on the scenery tree, all frames in the tree estimate threir position and orientation w.r.t. the root frame.
%
This way rays can easily and fast be transformed back and forth from the root tree to an individual object frame.
%-------------------------------------------------------------------------------
\section{How to set up a scenery in source code}
%
We will build a little scenery of a house with a roof and chimney as well as a simple tree. Further we add a small telescope with a reflective imaging mirror.
%
First we will define the geometry and their surfaces, second we will declare the relations between them. Third and finally we will update all frames relation w.r.t. the root frame to enable fast tracing (post initializing).
% 
%
First we define the main frame of our scenery. The main frame, often called world, will be the root of the scenery tree \ref{SubSecRootFrame}. 
%
\CppFileStartEnd{examples/set_up_scenery.cpp}{world}{tree}
%
Second we define frames that hold individual structures like a tree which will be composed from several objects. The tree will be placed in $x=5\,$m w.r.t. its later mother frame, i.e. the wolrd. 
%
\CppFileStartEnd{examples/set_up_scenery.cpp}{tree}{house}
Also part of the tree is the wooden pole.
\CppFileStartEnd{examples/set_up_scenery.cpp}{leaf_ball}{tree_pole}
and the rest of the source...
\CppFileStartEnd{examples/set_up_scenery.cpp}{tree_pole}{end_set_up_scene_in_source}
%-------------------------------------------------------------------------------
\section{Numerical functions}
\newcommand{\la}{\lambda}
%
The \lstinline{Function::Func1D} class provides $1$D mapping for floating numbers.
%
\begin{eqnarray}
    y &=& f(x)\\
    x &\in& X
\end{eqnarray}
%
All functions have limits which need to be respected. 
%
Any call of a function $f(x)$ with $x \notin X$ will throw an exception. 
%
We are strict about this behaviour to enforce that no propagation passes silently where e.g. your mirror's reflective index is only defined up to $600\,$nm but you shoot $800\,$nm photons onto it. 
%
Functions live in their own namespace.
\CppFileStartEnd{examples/set_up_scenery.cpp}{using_namespace}{using_namespace_end}
%
\subsection{Function limits}
%
First we define limits to our functions.
%
\CppFileStartEnd{examples/set_up_scenery.cpp}{func_limits}{func_limits_assert}
%
The limits here include the lower bound $0.0$ and exclude the upper one $1.0$.
%
Let's see the acceptance of our limits.
%
\CppFileStartEnd{examples/set_up_scenery.cpp}{func_limits_assert}{func_limits_constant}
%
When we create functions with our limits like
%
\CppFileStartEnd{examples/set_up_scenery.cpp}{func_limits_constant}{func_limits_const_call}
%
the function will also show the restrictive access:
%
\CppFileStartEnd{examples/set_up_scenery.cpp}{func_limits_const_call}{func_limits_call_end}
%
\subsection{Constant}
\begin{eqnarray}
    y &=& f(x) = c
\end{eqnarray}
%
A constant function takes its single constant value e.g. $1.337$ and its limits.
%
\CppFileStartEnd{examples/set_up_scenery.cpp}{func_const}{func_const_call}
%
When called within the limits, it will always return its constant value.
%
\CppFileStartEnd{examples/set_up_scenery.cpp}{func_const_call}{func_const_call_end}
\CenFig{figures/function_const.png}
%
\subsection{Polynom3}
The versatile polynom of power 3 is defined by its four parameters $a,b,c$ and $d$.
\begin{eqnarray}
    y &=& f(x) = ax^3 + bx^2 + cx^1 + dx^0
\end{eqnarray}
%
We initialize it using $a,b,c,d$ and the limits. Here we create a linear mapping.
\CppFileStartEnd{examples/set_up_scenery.cpp}{func_poly3}{func_poly3_call}
\CenFig{figures/function_polynom1.png}
%
We can do a quadratic mapping:
\CppFileStartEnd{examples/set_up_scenery.cpp}{func_poly3_quad}{func_poly3_quad_end}
\CenFig{figures/function_polynom2.png}
%
\subsection{Concatenation}
\CppFileStartEnd{examples/set_up_scenery.cpp}{func_concat}{func_concat_end}
\CenFig{figures/function_concat.png}
%-------------------------------------------------------------------------------
\bibliography{mct}
%-------------------------------------------------------------------------------
\begin{acronym}
    \acro{mct}[MCt]{Monte Carlo Tracer}
\end{acronym}
\end{document}