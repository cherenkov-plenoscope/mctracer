\documentclass[11pt,a4paper,oneside,titlepage]{book}

\usepackage[english]{babel}
\usepackage{lineno,hyperref}
\usepackage[nolist]{acronym}
\usepackage[onehalfspacing]{setspace}
\usepackage{booktabs}
\usepackage{float}
\usepackage{amsmath}
\usepackage{listings}
\usepackage{color}
\usepackage{hyperref}
\usepackage{graphicx}
\usepackage{blindtext}

\definecolor{LightGray}{gray}{0.70}
\lstdefinestyle{MctCpp}{%
    language=C++,
    keywordstyle=\bfseries,
    commentstyle=\itshape,
    rangeprefix=//--,rangesuffix=--,
    includerangemarker=false,
    columns=spaceflexible,
    escapeinside={/*@}{@*/},
    tabsize=4,
    frame=leftline,
    rulecolor=\color{LightGray},
    basicstyle=\ttfamily,
    numbers=left,
    numberstyle=\normalfont\tiny\color{LightGray},
    xleftmargin=0.75cm,
}

\lstdefinestyle{MctTxt}{%
    language={},
    keywordstyle=\bfseries,
    commentstyle=\itshape,
    rangeprefix=//--,rangesuffix=--,
    includerangemarker=false,
    columns=spaceflexible,
    escapeinside={/*@}{@*/},
    tabsize=4,
    frame=leftline,
    rulecolor=\color{LightGray},
    basicstyle=\ttfamily,
    numbers=left,
    numberstyle=\normalfont\tiny\color{LightGray},
    xleftmargin=0.75cm,
}


%\oddsidemargin = 31pt
%\topmargin = 20pt
%\headheight = 12pt
%\headsep = 25pt
\textheight = 25cm
\textwidth = 15cm
%\marginparsep = 10pt
%\marginparwidth = 35pt
%\footskip = 30pt

%\marginparpush = 7pt (not shown)
\hoffset = -1cm
\voffset = -2cm
%\paperwidth = 597pt
%\paperheight = 845pt


%\modulolinenumbers[1]

\begin{document}
%------------------------------------------------------------------------------
\newcommand{\CppFileStartEnd}[3]{%
    \begin{footnotesize}
        \lstinputlisting[linerange=#2-#3, style=MctCpp]{#1}%
    %\hfill\path{#1}\\
    \end{footnotesize}
}

\newcommand{\TxtFile}[1]{%
    \begin{footnotesize}
        \lstinputlisting[style=MctTxt]{#1}%
    \end{footnotesize}
}

\newcommand{\ill}[1]{% In Line Listing
    \begin{lstlisting}#1\end{lstlisting}    
}

\newcommand{\CenFig}[2]{
    \begin{figure}[H]
        \begin{center}
            \includegraphics[width=#2\textwidth]{#1}
        \end{center}
    \end{figure}
}

\makeatletter
\lst@Key{matchrangestart}{f}{\lstKV@SetIf{#1}\lst@ifmatchrangestart}
\def\lst@SkipToFirst{%
    \lst@ifmatchrangestart\c@lstnumber=\numexpr-1+\lst@firstline\fi
    \ifnum \lst@lineno<\lst@firstline
        \def\lst@next{\lst@BeginDropInput\lst@Pmode
        \lst@Let{13}\lst@MSkipToFirst
        \lst@Let{10}\lst@MSkipToFirst}%
        \expandafter\lst@next
    \else
        \expandafter\lst@BOLGobble
    \fi}
\makeatother
%------------------------------------------------------------------------------
\newcommand{\tool}{mctracer}
\newcommand{\thetitle}{\tool\\ \large{photon propagation in complex sceneries}}
%
\thispagestyle{empty}
\begin{center}
\Huge\textbf{\thetitle}
%
\vfill
%
\Large
mindset and how to use\\
\vspace{20pt}
\normalsize
{by\\\Large Sebastian Achim M\"uller } \\[5pt]
%
{\normalsize sebmuell@phys.ethz.ch}\\
%
\vspace{20pt}
Institute for Particle Physics
\par\smallskip\noindent
ETH Zurich\\2015-2017
\end{center}
%-------------------------------------------------------------
\pagenumbering{Roman}
\addcontentsline{toc}{chapter}{Contents}
\tableofcontents
%\newpage
%\addcontentsline{toc}{chapter}{Figures}
%\listoffigures
%\newpage
%\addcontentsline{toc}{chapter}{Tables}
%\listoftables
%-------------------------------------------------------------
\cleardoublepage
%
\setcounter{page}{0}
\pagenumbering{arabic}
%
\chapter{Abstract}
%
The \tool{} is a simulation for geometrical optics. 
%
It can propagate photons in a complex 3D scenery.
%
The \tool{} simulates reflection, refraction and absorbtion.
%
It does not cover diffraction.
%
For the investigation of optical devices or phenomena, \tool{} records the full photon's trajectory starting with the production, through all the photon's interactions until its final absorbtion.
%
A small set of primitiv surfaces is provided in \tool{} to form simple optical devices, such as lenses, imaging mirrors, light concentrators and aperture stops.
%
Further, complex objects can be simulated using triangular meshes.
%
To produce photons, \tool{} comes with a set of light sources to illuminate your scenery.
%
For more complex light sources, photons can be read from external files.
%
\tool{} can handle very complex sceneries with million of primitives while beeing fast and accurate on a scientific level.
%
You can feed your scenery into \tool{} using common CAD files for triangular meshes and a custom \tool{} xml file which describes the primitives provided by \tool{} itself.
%
When the provided tools do not cover your demand you have the chance to implement the features yourself.   
%
The \tool{} was originally created for simulations in Astro Particle Physics.
%
Imaging Atmospheric Cherenkov Telescopes like FACT and the CTA MST made use of \tool{} to investigate and improve their performance.
%
The \acf{acp} was born in this simulation.
%
\chapter{The \acf{acp}}
%
The \tool{} can simulate \acp{acp}.
%
An \ac{acp} consists out of two main parts.
%
First, an imaging system like e.g. a segmented imaging reflector as it is used for classic \acp{iact}.
%
Second, a light field sensor.
%
%------------------------------------------------------------------------------
\section{Create an \ac{acp} scenery}
%
Lets create an example scenery of an \ac{acp} called PLERITAS, which is a plenoptic extension to the VERITAS \ac{iact}.
%
\begin{lstlisting}[style=MctBash]
/demo$ mkdir pleritas
/demo$ cd pleritas/
\end{lstlisting}
All the resources needed to describe PLERITAS have to be in a folder.
%
\begin{lstlisting}[style=MctBash]
demo/pleritas$ vi scenery.xml
\end{lstlisting}
%
Create a xml file called scenery.xml and describe your scenery in there.
%
For our PLERITAS we use a basic VERITAS like imaging reflecor (created using the segmented reflector tool),\\
%
\XmlFileStartEnd{demo/pleritas/scenery.xml}{reflector_s}{stl_s}
%
a light field sensor\\
%
\XmlFileStartEnd{demo/pleritas/scenery.xml}{light_field_sensor_s}{light_field_sensor_e}
%
and an additional spider web which causes additional shadowing and is described in a CAD file.\\
%
\XmlFileStartEnd{demo/pleritas/scenery.xml}{stl_s}{light_field_sensor_s}
%
\newline
\SideBySide{
\CenFig{demo/pleritas_reflector.png}{1.0}	
}{
\CenFig{demo/pleritas_light_field_sensor.png}{1.0}	
}
%
\SideBySide{
\CenFig{demo/pleritas_focus_reflector.png}{1.0}	
}{
\CenFig{demo/sensor_back.png}{1.0}	
}
%
All resources like the CAD file of the spider web must be placed in the scenery folder.
%
\begin{lstlisting}[style=MctBash]
/demo/pleritas$ ls
scenery.xml  spider.stl
\end{lstlisting}
%

Explore your scenery using mctShow.
%
\begin{lstlisting}[style=MctBash]
/demo/pleritas$ mctShow -s scenery.xml
\end{lstlisting}
%
%------------------------------------------------------------------------------
\section{Run the light field calibration}
%
\begin{lstlisting}[style=MctBash]
/demo$ mctPlenoscopeCalibration -i scenery -o pleritas_calibration -n 3
Plenoscope Calibrator: propagating 3M photons
1 of 3
2 of 3
3 of 3
/demo$ 
\end{lstlisting}
%
\CppFileStartEnd{../Plenoscope/Calibration/LixelStatistics.h}{lixel_statistics_s}{lixel_statistics_e}
%
%------------------------------------------------------------------------------
\section{Simulate \ac{acp} responses to \ac{eas}}
%
\begin{lstlisting}[style=MctBash]
/demo$ mctPlenoscopePropagation -c propagation_config.xml -l pleritas_calibration -i /some/simulation/gamma1_RUN41.dat -o my_run
event 1, PRMPAR 1, E 225.569 GeV
event 2, PRMPAR 1, E 168.365 GeV
event 3, PRMPAR 1, E 46.5717 GeV
...
\end{lstlisting}
%------------------------------------------------------------------------------
\section{Explore the siumlated events}

\section{Light Field calibration}
%
Each read out channel (lixel) on the light field sensor of the \ac{acp} corresponds to a specific ray in the light field. 
%
Each of these rays has a support on the principal aperture plane at position $x$ and $y$ on and a direction vector descibed by $c_x$ and $c_y$.
%
Further, each of these lixel has a specific time delay $t_\text{delay}$ which the light needs to travel when comig from the principal aperture plane and each lixel has its own efficency $\eta$ due to its specific geometrical position in the set up.
%
So in the light field calibration we determine $\eta$, $x$, $y$, $c_x$, $c_y$ and $t_\text{delay}$ for each lixel.
%
The calibration is done by throughing photons into the \ac{acp}, where we randomly draw both the photons intersection on the principal aperture plane $x$, $y$ and their incoming direction $c_x$, $c_y$.
%
In the calibration, many photons are used and several of them will be absorbed in the lixels.
%
For each lixel, there is list of the photon properties ($x$, $y$, $c_x$, $c_y$ and $t_\text{delay}$) of the photons that reached this lixel.
%
From this list, the lixel is assigned the averages of all these properties, as well as their standard deviations.
%
The number of photons reaching the lixel during the calibration gives the efficiency $\eta$.
\chapter{1D functions}
\newcommand{\la}{\lambda}
%
The \lstinline{Function::Func1D} class provides $1$D mapping for floating numbers.
%
\begin{eqnarray}
    y &=& f(x)\\
    x &\in& X
\end{eqnarray}
%
All functions have limits which need to be respected. Any call of a function $f(x)$ with $x \notin X$ will throw an exception. We are strict about this behaviour to enforce that no propagation passes silently where e.g. your mirror's reflective index is only defined up to $600\,$nm but you shoot $800\,$nm photons onto it. Functions live in their own namespace.
%
\CppFileStartEnd{examples/Func1DExample.cpp}{using_namespace}{using_namespace_end}
%------------------------------------------------------------------------------
\section{Domains and their limits}
%
First we define limits for the domains of our functions.
%
\CppFileStartEnd{examples/Func1DExample.cpp}{func_limits}{func_limits_assert}
%
The limits here include the lower bound $0.0$ and exclude the upper one $1.0$. A limit can assert that a given argument is in its domain. If not, it will throw an exception.
%
\CppFileStartEnd{examples/Func1DExample.cpp}{func_limits_assert}{func_limits_constant}
%
All functions have a domain within their limits. The limits are given to the funcions during construction.
%
\CppFileStartEnd{examples/Func1DExample.cpp}{func_limits_constant}{func_limits_const_call}
%
Functions assert, the argument to be inside their domain.
%
\CppFileStartEnd{examples/Func1DExample.cpp}{func_limits_const_call}{func_limits_call_end}
%------------------------------------------------------------------------------
\section{Constant}
%
Sometimes it needs a constant function which will return the same value for any argument inside their domain limits.
%
\begin{eqnarray}
    y &=& f(x) = c
\end{eqnarray}
%
A constant function is created given its single constant value e.g. $1.337$ and its domain limits.
%
\CppFileStartEnd{examples/Func1DExample.cpp}{func_const}{func_const_call}
%
When called, within the limits, it will always return its constant value.
%
\CppFileStartEnd{examples/set_up_scenery.cpp}{func_const_call}{func_const_call_end}
\CenFig{figures/function_const.png}{0.75}
%------------------------------------------------------------------------------
\section{Polynom3}
%
The versatile polynom to the power of 3 is defined by its four parameters $a,b,c$ and $d$.
%
\begin{eqnarray}
    y &=& f(x) = ax^3 + bx^2 + cx^1 + dx^0
\end{eqnarray}
%
We initialize the \lstinline{Polynom3} using $a,b,c,d$ and the limits. 
%
By setting the higer orders to zero, we create e.g. a linear mapping.
%
\CppFileStartEnd{examples/Func1DExample.cpp}{func_poly3}{func_poly3_call}
\CenFig{figures/function_polynom1.png}{0.75}
%
We can do a quadratic mapping.
\CppFileStartEnd{examples/Func1DExample.cpp}{func_poly3_quad}{func_poly3_quad_end}
\CenFig{figures/function_polynom2.png}{0.75}
%
The full polynom to the power of $3$.
%
\CppFileStartEnd{examples/Func1DExample.cpp}{func_poly3_tri}{func_poly3_tri_end}
\CenFig{figures/function_polynom3.png}{0.75}
%
%------------------------------------------------------------------------------
\section{Linear interpolation look up table}
%
In some cases, it can be tough to model an analytic $1D$ function. In these cases one can still use the a look up table with linear interpolation.
%
The input table also defines the domain limits, so no limits have to be given during construction.
%
\CppFileStartEnd{examples/Func1DExample.cpp}{look_up}{look_up_end}
\CenFig{figures/function_interpol.png}{0.75}
%------------------------------------------------------------------------------
\section{Concatenation}
%
Functions can be concatenated when their domain limits match.
%
The functions to be concatenated can be of any kind, even concatenated functions themselve.
%
Since the concatenated function can deduce its domain limits from the input functions, no limit has to be given during construction.
%
\CppFileStartEnd{examples/Func1DExample.cpp}{func_concat}{func_concat_end}
\CenFig{figures/function_concat.png}{0.75}
%------------------------------------------------------------------------------
\section{Access}
%
Access to the values of a function is done using the bracket operator.
%
\CppFileStartEnd{examples/Func1DExample.cpp}{func_access}{func_access_end}
%
Also a function can provide a table of both argument and value. The number of samples along the domain limits of the function can be specified.
% 
\CppFileStartEnd{examples/Func1DExample.cpp}{func_access_sampling}{func_access_sampling_end}
%
Using the ascci io, a function can be exported into a text file.
%
\CppFileStartEnd{examples/Func1DExample.cpp}{func_access_sampling_export}{func_access_sampling_export_end}
%
The output text file is a two column matrix. First column is the argument $x$, second is the function value $f(x)$.
%
\TxtFile{figures/my_p3.txt}
%------------------------------------------------------------------------------
%-------------------------------------------------------------------------------
\chapter{How to set up a scenery in source code}
%
We will build a little scenery of a house with a roof and chimney as well as a simple tree. Further we add a small telescope with a reflective imaging mirror.
%
First we will define the geometry and their surfaces, second we will declare the relations between them. Third and finally we will update all frames relation w.r.t. the root frame to enable fast tracing (post initializing).
% 
%
First we define the main frame of our scenery. The main frame, often called world, will be the root of the scenery tree \ref{SubSecRootFrame}. 
%
\CppFileStartEnd{../Tests/Examples/set_up_scenery.cpp}{world}{tree}
%
Second we define frames that hold individual structures like a tree which will be composed from several objects. The tree will be placed in $x=5\,$m w.r.t. its later mother frame, i.e. the wolrd. 
%
\CppFileStartEnd{../Tests/Examples/set_up_scenery.cpp}{tree}{house}
Also part of the tree is the wooden pole.
\CppFileStartEnd{../Tests/Examples/set_up_scenery.cpp}{leaf_ball}{tree_pole}
and the rest of the source...
\CppFileStartEnd{../Tests/Examples/set_up_scenery.cpp}{tree_pole}{end_set_up_scene_in_source}
%------------------------------------------------------------------------------
%--references--
\renewcommand{\bibname}{References}
\bibliography{mct}
\bibliographystyle{plain}  
\addcontentsline{toc}{chapter}{\bibname}
%
\begin{acronym}
    \acro{mct}[MCt]{Monte Carlo Tracer}
    \acro{acp}[ACP]{Atmospheric Cherenkov Plenoscope}
    \acro{iact}[IACT]{Imaging Atmospheric Cherenkov Telescope}
    \acro{eas}[EAS]{Extensive Atmospheric Showers}
\end{acronym}
%--Appendix--
\end{document}