\documentclass[review]{elsarticle}

\usepackage{lineno,hyperref}
\usepackage[nolist]{acronym}
\usepackage[onehalfspacing]{setspace}
\usepackage{booktabs}
\usepackage{float}
\usepackage{amsmath}
\usepackage{listings}
\usepackage{color}
\usepackage{hyperref}

\lstdefinestyle{MctCpp}{%
    language=C++,
    keywordstyle=\bfseries,
    commentstyle=\itshape,
    rangeprefix=//--,rangesuffix=--,
    includerangemarker=false,
    columns=spaceflexible,
    escapeinside={/*@}{@*/},
    tabsize=4,
    frame=leftline,
    basicstyle=\ttfamily,
    numbers=left
}

\modulolinenumbers[1]
\journal{Journal of \LaTeX\ Templates}

%%%%%%%%%%%%%%%%%%%%%%%
%% Elsevier bibliography styles
%%%%%%%%%%%%%%%%%%%%%%%
%% To change the style, put a % in front of the second line of the current style and
%% remove the % from the second line of the style you would like to use.
%%%%%%%%%%%%%%%%%%%%%%%

%% Numbered
%\bibliographystyle{model1-num-names}

%% Numbered without titles
%\bibliographystyle{model1a-num-names}

%% Harvard
%\bibliographystyle{model2-names.bst}\biboptions{authoryear}

%% Vancouver numbered
%\usepackage{numcompress}\bibliographystyle{model3-num-names}

%% Vancouver name/year
%\usepackage{numcompress}\bibliographystyle{model4-names}\biboptions{authoryear}

%% APA style
%\bibliographystyle{model5-names}\biboptions{authoryear}

%% AMA style
%\usepackage{numcompress}\bibliographystyle{model6-num-names}

%% `Elsevier LaTeX' style
\bibliographystyle{elsarticle-num}
%%%%%%%%%%%%%%%%%%%%%%%

\begin{document}
%-------------------------------------------------------------------------------
\newcommand{\CppFileStartEnd}[3]{%
    \begin{footnotesize}
        \lstinputlisting[linerange=#2-#3, style=MctCpp]{#1}%
    \hfill\path{#1}\\
    \end{footnotesize}
}

\makeatletter
\lst@Key{matchrangestart}{f}{\lstKV@SetIf{#1}\lst@ifmatchrangestart}
\def\lst@SkipToFirst{%
    \lst@ifmatchrangestart\c@lstnumber=\numexpr-1+\lst@firstline\fi
    \ifnum \lst@lineno<\lst@firstline
        \def\lst@next{\lst@BeginDropInput\lst@Pmode
        \lst@Let{13}\lst@MSkipToFirst
        \lst@Let{10}\lst@MSkipToFirst}%
        \expandafter\lst@next
    \else
        \expandafter\lst@BOLGobble
    \fi}
\makeatother
%-------------------------------------------------------------------------------

\begin{frontmatter}

\title{%
	The Monte Carlo Tracer%\tnoteref{mytitlenote}
}
%\tnotetext[mytitlenote]{%
%	Fully documented templates are available in the elsarticle package on \href{http://www.ctan.org/tex-archive/macros/latex/contrib/elsarticle}{CTAN}.
%}

%% Group authors per affiliation:
%\author{The FACT collaboration}%\fnref{myfootnote}}
%\address{Radarweg 29, Amsterdam}
%\fntext[myfootnote]{TU Dortmund, ETH Zuerich}

%% or include affiliations in footnotes:
\author[mymainaddress,mysecondaryaddress]{Sebastian Achim Mueller}
%\ead[url]{www.fact-project.org}

\cortext[mycorrespondingauthor]{Sebastian Achim Mueller}
\ead{sebmuell@phys.ethz.ch}

\address[mymainaddress]{Institute for Particle Physics, ETH, Otto-Stern-Weg 5, 8093 Zuerich, Switzerland}
\address[mysecondaryaddress]{Experimental Physics 5b, TU Dortmund, Otto-Hahn-Strasse 4, 44227 Dortmund, Germany}
%-------------------------------------------------------------------------------
\begin{abstract}
%
In $\gamma$ ray and cosmic ray astronomy it needs dedicated simulations of detectors to develope and run the instruments which observe particle interactions far beyond any energy accessable in the lab.
%
The Monte Carlo Tracer exists since we do what we must, because we can.
%
\end{abstract}
%-------------------------------------------------------------------------------
\begin{keyword}
ray tracing, photon propagation
%
\end{keyword}
%-------------------------------------------------------------------------------
\end{frontmatter}
%\linenumbers
%-------------------------------------------------------------------------------
\section{The scenery tree}
%-------------------------------------------------------------------------------
\subsection{The root of the scenery tree}
\label{SubSecRootFrame}
The frame at the root of the tree structure represents the whole scenery.
%
Before ray tracing is performed on the scenery tree, all frames in the tree estimate threir position and orientation w.r.t. the root frame.
%
This way rays can easily and fast be transformed back and forth from the root tree to an individual object frame.
%-------------------------------------------------------------------------------
\section{How to set up a scenery in source code}
%
First we define the main frame of our scenery. This frame, often called world, will be the root of the scenery \ref{SubSecRootFrame}. 
%
\CppFileStartEnd{examples/set_up_scenery.cpp}{world}{tree}
%
Second we define frames that will hold individual structures like a tree which will be composed from several objects. The tree will be placed in $x=5\,$m w.r.t. its later mother frame, i.e. the wolrd. 
%
\CppFileStartEnd{examples/set_up_scenery.cpp}{tree}{leaf_ball}
Also part of the tree is the wooden pole.
\CppFileStartEnd{examples/set_up_scenery.cpp}{leaf_ball}{tree_pole}
%-------------------------------------------------------------------------------
\bibliography{mct}
%-------------------------------------------------------------------------------
\begin{acronym}
    \acro{mct}[MCt]{Monte Carlo Tracer}
\end{acronym}
\end{document}